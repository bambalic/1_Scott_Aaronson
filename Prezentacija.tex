\documentclass{beamer}
\usepackage{beamerthemeshadow}
\usepackage{graphicx}
\usepackage{color}
\usepackage[utf8]{inputenc}
\usepackage{hyperref}
\usepackage[flushleft]{threeparttable}
\definecolor{beamer@green}{rgb}{0.5, 2, 0.5}
\setbeamercolor{structure}{fg=beamer@green}

\def\dJ{{\fontencoding{T1}\selectfont\dj}}
\def\Dj{{\fontencoding{T1}\selectfont\DJ}}

\begin{document}
\title{Scott Aaronson}
\author{Vidak Kozomara, Lazar Perišić, Aleksa Cvetković, \newline \Dj{}or\dJ{}e Milošević}
\institute{Matematički fakultet\\Univerzitet u Beogradu}
\date{
	\footnotesize{Beograd, 2019.}	
}

\begin{frame}
	\thispagestyle{empty}
	\titlepage
\end{frame}

\addtocounter{framenumber}{-1}
%-----------------------------------------------------------------
\section{Biografija}
\subsection{Kratak uvod}
\begin{frame}[fragile]
    \frametitle{Biografija}
	\begin{itemize}
		\item \textbf{Skot Džoel Aronson} (eng. \textit{Scott Joel Aaronson}) je američki teorijski informatičar i profesor informatike na Univerzitetu u Ostinu, Teksas. Njegove primarne oblasti istraživanja su mogućnosti i limiti kvantnih računara kao i računarska teorija složenosti.
	\end{itemize}
\end{frame}
%------------------------------------------------------------------------
\section{Mladost i obrazovanje}
\begin{frame}
	\frametitle{Mladost i obrazovanje} 
%	\tableofcontents[hidesubsections]
	\begin{itemize}
	    \item Ro\dJ{}en u Filadelfiji, SAD
	    \item Boravak u Aziji
	    \item Dana Moškovic
	    \item Kornel Univerzitet
	    \item Berkli Univerzitet
	    
	\end{itemize}
\end{frame}
%-----------------------------------------------------------------------


\begin{frame}[fragile]\frametitle{Kako ljudi pamte?}
	\begin{itemize}	
		\item 10\% onoga što pročitaju
		\item 20\% onoga što čuju
		\item 30\% onoga što vide
		\item 50\% onoga što čuju i vide
		\item 70\% onoga što kažu i napišu
		\item 90\% onoga što rade
	\end{itemize}
\end{frame}

\begin{frame}[fragile]\frametitle{Kako ljudi uče?}
	\begin{itemize}	
		\item Verbalnim putem (čitanjem i slušanjem)
		\item Vizuelnim putem (gledanjem)
		\item Aktivnim učešćem (gledanjem, slušanjem i činjenjem)
	\end{itemize}
\end{frame}

\begin{frame}[fragile]\frametitle{Kako ljudi uče?}
	\begin{itemize}	
		\item Ljudi ne komuniciraju samo rečima. Prema nekim
		istraživanjima:
		\begin{itemize}
			\item samo 8\% poruke se prenese samim rečima (verbalna komunikacija)
			\item 37\% se prenesi bojom glasa, tonalitetom, pauzama u govoru (paralingvističkim znakovima)
			\item 55\% poruke se prenosi govorom tela: pratećim pokretima, izrazom lica i
			očiju, stavom tela i drugo (neverbalna komunikacija)
		\end{itemize}
		\item Verbalnim putem se najčešće prenose činjenice i
		sirove informacije, dok se neverbalnim putem prenose stavovi i
		emocionalni odnos prema činjenicama koje izlažemo.
	\end{itemize}
\end{frame}


\begin{frame}[fragile]\frametitle{Opšte preporuke za dobru usmenu prezentaciju}
	\begin{itemize}	
		\item Priprema
		\begin{itemize}
			\item Jako je važno dobro se pripremiti za izlaganje i provežbati prezentaciju kod kuće
			\item Poruku koju želite da prenesete prilagodite Vašoj publici -- ne treba da bude isto izlaganje za studente i izlaganje na nekom poslovnom sastanku
			\item Dobro koncipirajte slajdove -- materijal koji se prezentuje treba da bude koncizan,
			sadržajan i zanimljiv
			\item Koristite vizuelna pomagala (fotografije, tablice, grafikone) -- štede vreme, bude radoznalost i pojačavaju
			utisak. ,,Jedna slika govori više od hiljadu reči.``
			\item Napravite dobar izbor boja (ne više od tri). Obratite pažnju na
			kontrast boje slova i pozadine
			\item Odaberite odgovarajuće veličine slova koje treba da odgovaraju važnosti inforamcija kao i veličini prostorije u kojoj se prezentacija održava			
		\end{itemize}		
	\end{itemize}
\end{frame}

\begin{frame}[fragile]\frametitle{Opšte preporuke za dobru usmenu prezentaciju}
	\begin{itemize}	
		\item Priprema (nastavak)
		\begin{itemize}		
			\item Proverite tehničke detalje, na primer, da li se
			prezentacija otvara na računaru i lepo prikazuje na ekranu
			\item Ne čekajte poslednji trenutak da bi pripremili slajdove
			\item Vežbajte pred ogledalom i merite vreme
			\item Ponovo vežbajte
		\end{itemize}		
	        \item Za dobru prezentaciju, Vi ste jednako bitni koliko i slajdovi
	        \item Obucite se profesionalno i udobno
			\item Proverite da li Vam je isključen mobilni telefon
			\item Započnite prezentaciju osmehom
			\item Uspostavite i održavajte kontakt očima
			\item Koristite pauze i mirnoću u glasu
			\item Menjajte brzinu i tonalitet govora
			\item Izbegavajte suviše kretanja kako ne bi iritirali
			publiku, ali nemojte ni stajati u mestu
	\end{itemize}
\end{frame}

\begin{frame}[fragile]\frametitle{Opšte preporuke za dobru usmenu prezentaciju}
	\begin{itemize}		
		\item Nikada nemojte čitati ceo tekst sa papira ili sa ekrana
		\item Trudite se da nemate konfuzne izjave. Izbegavajte suprotnosti
				\item Podaci u prezentaciji moraju teći logičkim sledom
		-- na slajdove pišite samo nužne informacije, ostalo dodajte usmenim izlaganjem
     	\item Neka prelazi iz jedne na drugu temu budu ,,glatki``		
		\item Obratite pažnju na vremensko trajanje prezentacije; ako prezentacija traje predugo, i vi i publika gubite koncentraciju
		\item Budite spremni da podelite sa publikom ono što znate ali
		i da čujete ono što ne znate
		\item Pažljivo saslušajte komentare publike. Pokušajte
		da objasnite publici ono što ste želeli da kažete		
	\end{itemize}
\end{frame}

\begin{frame}[fragile]\frametitle{Struktura prezentacije}
	\begin{itemize}	
		\item  Uvod
		\begin{itemize}
			\item Pozdravite publiku
			\item Pohvalite organizaciju doga\d{}aja
			\item Pokažite dobre manire
		\end{itemize}
		\item Početak
		\begin{itemize}
			\item Ako je moguće, pokušajte da se povežete sa prethodnim govornikom
			\item Postavite ,,okvir`` prezentacije
			\item Privucite pažnju nekom ubedljivom pričom ili
			anegdotom
			\item Definišite glavne teme, ne više od tri
		\end{itemize}		
	\end{itemize}
\end{frame}

\begin{frame}[fragile]\frametitle{Struktura prezentacije}
	\begin{itemize}			
		\item Središnji deo
		\begin{itemize}
			\item Prikažite kratko šta su drugi uradili na temu koju
			prezentujete
			\item Prikažite svoje rezultate i doprinose
			\item Nemojte samo prikazivati sirove formule. Pokušajte da ih nekako
			obrazložite
		\end{itemize}
		\item Kraj
		\begin{itemize}
			\item Rekapitulacija. Ponovite glavne teme
			\item Ponovite ključne rezultate
			\item Navedite literaturu koju ste koristili u istraživanju
			\item Poslednjim slajdom zahvalite publici na pažnji		
		\end{itemize}		
	\end{itemize}
\end{frame}



\begin{frame}[fragile]\frametitle{Kreiranje slajdova i prezentacija}
	\begin{itemize}	
		\item \LaTeX{} se može koristiti za kreiranje atraktivnih slajdova i prezentacija	
		\item Osnovna podrška za kreiranje slajdova u \LaTeX{}-u postoji u vidu klase \verb"slides"		
		\item Ova klasa podrazumeva korišćenje velikih, bez-serifnih slova čime se dobija dokument koji je pogodan za prikazivanje na projektoru		
		\item Pored ovoga, klasa \verb"slides" ne pruža nikakvu dodatnu podršku za kreiranje prezentacija		
		\item \textcolor{beamer@darkred}{Primer 13}
	\end{itemize}
\end{frame}

\begin{frame}[fragile]\frametitle{Napredna sredstva za kreiranje prezentacija}
	\begin{itemize}	
		\item Namenske komande koje olakšavaju rad sa slajdovima mogu se naći u dodatnim \LaTeX{} paketima		
		\item Postoji veći broj takvih paketa sa sličnim mogućnostima		
		\item Ovde će biti predstavljen paket \verb"beamer" koji ima odličnu podršku za kreiranje
		dinamičkih prezentacija \footnotesize (\url{http://tug.ctan.org/macros/latex/contrib/beamer/doc/beameruserguide.pdf}) \normalsize		
		\item Predvi\d{}eno je da se dokumenti kreirani uz pomoć ovog paketa prevode u pdf
		format radi prikazivanja prezentacije na ekranu, pri čemu
		\verb"beamer" nudi čitav niz efekata za postupni prikaz sadržaja, kao i za prelaz izme\d{}u dva slajda		
		\item Paket omogućava i generisanje slajdova za prikaz na
		projektoru, ili generisanje štampane verzije prezentacije
	\end{itemize}
\end{frame}

\begin{frame}[fragile]\frametitle{Napredna sredstva za kreiranje prezentacija}
	\begin{itemize}	
		\item Paket \verb"beamer" sadrži definiciju istoimene klase dokumenata		
		\item \LaTeX{} dokumenti koji predstavljaju prezentacije kreirane korišćenjem ovog paketa treba da
		počnu sa:
		
		\verb"\documentclass[opcije]{beamer}"
		
		\item Stil prezentacije je odre\d{}en tzv. temom, koja se u preambuli dokumenta
		zadaje komandom: \verb"\usetheme{tema}"		
		\item Tema odre\d{}uje boju pozadine i teksta, fontove kojima će biti
		ispisani naslovi ili običan tekst, grafiku koja će biti prikazana na svakom slajdu
		i tako dalje		
		\item Na raspolaganju je veliki broj podrazumevanih tema, koje su nazvane po gradovima,
		na primer \verb"Antibes", \verb"Berlin", \verb"Copenhagen", \verb"Frankfurt", \verb"Madrid", \verb"Szeged" i \verb"Warsaw"
	\end{itemize}
\end{frame}

\begin{frame}[fragile]\frametitle{Napredna sredstva za kreiranje prezentacija}
	\begin{itemize}	
		\item	Navedene teme u potpunosti odre\d{}uju stil prezentacije, a 
		pojedinačni aspekti prezentacije se mogu kontrolisati tzv. podtemama, 
		koje se mogu svrstati u četiri kategorije navedene u tabeli: 		
		\begin{tabular}{|l|l|}
			\hline	
			\footnotesize	komanda & \footnotesize značenje\\
			\hline
			\hline \footnotesize
			\footnotesize   \verb"\useoutertheme{podtema}" &\footnotesize kontroliše dekoracije na slajdovima \\
			\footnotesize	\verb"\useinnertheme{podtema}" &\footnotesize kontroliše izgled glavnog dela na slajdovima \\
			\footnotesize	\verb"\usefonttheme{podtema}" &\footnotesize kontroliše fontove na slajdovima \\
			\footnotesize	\verb"\usecolortheme{podtema}" &\footnotesize kontroliše boje na slajdovima\\
			\hline
		\end{tabular}
	\end{itemize}
\end{frame}

\begin{frame}[fragile]\frametitle{Napredna sredstva za kreiranje prezentacija}
	\begin{itemize}	
		\item Još finija kontrola nad pojedinim aspektima prezentacije ostvaruje se komandama
		\verb"\setbeamertemplate", \verb"\setbeamerfont" i \verb"\setbeamercolor"		
		\item	Tako se na primer ikonice za navigaciju kroz prezentaciju, koje bivaju automatski generisane u svakoj
		\verb"beamer" prezentaciji, eliminišu komandom: \verb"\setbeamertemplate{navigation symbols}{}"		
		\item	Klasa \verb"beamer" redefiniše neke standardne \LaTeX{} komande  		
	\end{itemize}
\end{frame}

\begin{frame}[fragile]\frametitle{Napredna sredstva za kreiranje prezentacija}
	\begin{itemize}	
			\item	Komande koje se mogu navesti u preambuli dokumenta date su u tabeli:			
			\begin{tabular}{|l|l|}
				\hline	
				\footnotesize	komanda & \footnotesize značenje\\
				\hline
				\hline \footnotesize
				\footnotesize \verb"title" &\footnotesize naslov prezentacije \\
				\footnotesize \verb"subtitle" &\footnotesize podnaslov prezentacije \\
				\footnotesize \verb"author" &\footnotesize autor (odnosno autori) prezentacije \\
				\footnotesize \verb"institute" &\footnotesize ime institucije sa koje dolazi autor\\
				\footnotesize \verb"date"&\footnotesize datum\\
				\hline
			\end{tabular}
			\item Komanda \verb"\titlepage" na osnovu vrednosti zadatih u preambuli dokumenta kreira 
			naslov prezentacije unutar datog slajda
	\end{itemize}
\end{frame}

\begin{frame}[fragile]\frametitle{Napredna sredstva za kreiranje prezentacija}
	\begin{itemize}	
		\item Pojedinačni slajdovi u dokumentu se navode unutar okruženja \verb"frame"		
			\item Ovo okruženje počinje komandom:		
		\verb"\begin{frame}{naslov}"
			
			a završava se komandom:
			
			\verb"\end{frame}"		
		\item	Argument \verb"naslov" u komandi kojom počinje okruženje predstavlja nisku koja će
		biti ispisana kao naslov slajda
	\end{itemize}
\end{frame}

\begin{frame}[fragile]\frametitle{Napredna sredstva za kreiranje prezentacija}
	\begin{itemize}	
		\item Različiti efekti prelaza sa jednog slajda na drugi mogu se postići stavljanjem odgovarajućih komandi unutar
		\verb"frame" okruženja 		
		\item	Neke od tih komandi su:
		\begin{itemize}
			\item  \verb"\transdissolve" --- tekući slajd se preliva u naredni slajd
			\item \verb"\transwipe" --- linija ,,briše`` ekran otkrivajući naredni slajd ili
			\item \verb"\transboxout" --- naredni slajd se pomalja preko tekućeg počev od centralnog dela slajda prema
			ivicama
		\end{itemize}		
		\item	Podrazumevani efekat je da naredni slajd neposredno zamenjuje tekući slajd 		
		\item	Trajanje efekta se može precizirati \verb"\transduration" komandom unutar \verb"frame" okruženja
	\end{itemize}
\end{frame}

\begin{frame}[fragile]\frametitle{Napredna sredstva za kreiranje prezentacija}
	\begin{itemize}	
		\item Unutar okruženja
		\verb"frame"
		mogu se koristiti sve \LaTeX{} komande za rad sa
		tekstom		
		\item	Na slajdovima se često koristi okruženje \verb"itemize"		
		\item	Izgled takozvanih \verb"bullet"-a, koji označavaju stavke liste na različitim nivoima
		hijerarhije, može se podešavati pomenutim komandama \verb"\setbeamertemplate", \verb"\setbeamerfont" i \verb"\setbeamercolor"
		\item \textcolor{beamer@darkred}{Primer 14}
	\end{itemize}
\end{frame}

\begin{frame}[fragile]\frametitle{Napredna sredstva za kreiranje prezentacija}
	\begin{itemize}	
		\item  Moguće je koristiti i animirane efekte za postupni prikaz sadržaja slajda		
		\item Ovakav efekat je najjednostavnije postići umetanjem komande \verb"\pause"
		na jednom ili više mesta unutar \verb"frame" okruženja		
		\item	Sadržaj slajda tada tokom prezentacije biva prikazan inkrementalno, i to prvo samo deo slajda do mesta gde je
		umetnuta prva komanda \verb"\pause", zatim se komandom za prelazak na naredni
		slajd u prezentaciji prikazuje i deo slajda do mesta gde je umetnuta naredna komanda
		\verb"\pause", i tako dalje.		
	\end{itemize}
\end{frame}

\begin{frame}[fragile]\frametitle{Napredna sredstva za kreiranje prezentacija}
	\begin{itemize}	
		\item Preciznija kontrola nad ovim efektom se može postići \verb"\onslide<lista>" komandom		
		\item Slajd se i u ovom slučaju prikazuje inkrementalno, te komanda za
		prelazak na naredni slajd u prezentaciji ovde aktivira deo po deo slajda		
		\item	Brojevi u listi koja se zadaje u \verb"\onslide" komandi označavaju u kom segmentu
		prikaza slajda će tekst koji sledi komandu biti vidljiv. 		
		\item	Brojevi u listi se razdvajaju zarezima, a niz brojeva je moguće kraće zapisati u obliku
		\verb"m-n", gde je \verb"m" prvi broj u nizu, a \verb"n" poslednji		
		\item	Ako se u \verb"\onslide" komandi lista izostavi, onda će tekst koji sledi biti vidljiv u svakom segmentu prikaza 
		datog slajda
	\end{itemize}
\end{frame}

\begin{frame}[fragile]\frametitle{Napredna sredstva za kreiranje prezentacija}
	\begin{itemize}	
		\item Ukoliko je na primer sadržaj slajda definisan na sledeći način:		
		\footnotesize
		\begin{verbatim}
			\onslide
			Suglasnici su: \\
			\onslide<1,2,3>b \\
			\onslide<2,3>c \\
			\onslide<3>d\\
			\onslide
			Samoglasnici su: \\
			\onslide<1-3>a \\
			\onslide<2-3>e \\
			\onslide<3>i
		\end{verbatim} \normalsize		
		tada će u prvom koraku prikaza slajda biti vidljiva slova \verb"b" i \verb"a", u drugom koraku
		će biti dodata slova \verb"c" i \verb"e", a u trećem koraku i slova \verb"d" i
		\verb"i", dok će tekst ,,Suglasnici/samoglasnici su:`` biti vidljiv sve vreme prikaza slajda
		\item \textcolor{beamer@darkred}{Primer 15}
	\end{itemize}
\end{frame}
\end{document}
